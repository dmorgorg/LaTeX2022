% !TEX root = ../../statikz2020/statikz2020.tex

%\DLOne[rotate]{tl}{tr}{b}{fill}{draw}{spaces}{scale}{lineWidth}
\newcommand{\DLDown}[9][0]{
	\def\rotate{#1}
	\def\tl{#2}
	\def\tr{#3}
	\def\b{#4}
	\def\lfill{#5}
	\def\ldraw{#6}
	\def\spaces{#7}
	\def\lscale{#8}
	\def\llinewidth{#9}

	\def\tocm{0.0351459804}
	\providecommand\tiplength{5pt}

	\gettikzxy{(\tl)}{\tlx}{\tly};
	\gettikzxy{(\tr)}{\trx}{\try};
	\gettikzxy{(\b)}{\bx}{\by};

	\pgfmathparse{\trx-\tlx} \let\dx\pgfmathresult
	\pgfmathparse{\try-\tly} \let\dy\pgfmathresult
	\pgfmathparse{round(\tocm*\dx*100)/100} \let\dxcm\pgfmathresult
	\pgfmathparse{round(\tocm*\dy*100)/100} \let\dycm\pgfmathresult

	\coordinate (midBase) at (\tlx/2+\trx/2, \by);

	\gettikzxy{(midBase)}{\mx}{\my}
	\pgfmathparse{round(\tocm*\mx*100)/100} \let\mxcm\pgfmathresult
	\pgfmathparse{round(\tocm*\my*100)/100} \let\mycm\pgfmathresult


	\pgfmathparse{\dxcm/\spaces} \let\xspacing\pgfmathresult
	\pgfmathparse{\dycm/\spaces} \let\yspacing\pgfmathresult
	% \numarrows is actually one less than the number of arrows since we start a 0
	\pgfmathparse{\spaces-1} \let\numarrows\pgfmathresult

	\begin{scope}[scale=\lscale, rotate around ={\rotate:(midBase)}, line cap = round, line join = round, line width = \llinewidth mm]
		\fill[\lfill] (\tlx, \tly) -- (\trx, \try) -- (\trx, \by) -- (\tlx, \by);
		\draw[\ldraw, thin] (\tlx, \tly) -- (\trx, \try) -- (\trx, \by) -- (\tlx, \by) -- cycle;

		\pgfmathparse{abs(\try - \by)} \let\ryy\pgfmathresult
		\pgfmathparse{abs(\tly - \by)} \let\lyy\pgfmathresult

		% \pgfmathparse{\lyy>\tiplength && \ryy>\tiplength}
		% \ifnum\pgfmathresult=1
		% 	\draw[\ldraw, {Latex[length=\tiplength]}-{Latex[length=\tiplength]},line width=\llinewidth] (\tlx,\by) -- (\tlx, \tly) -- (\trx,\try) -- (\trx,\by);
		% \fi
		% \pgfmathparse{\lyy<\tiplength && \lyy>\tiplength/4 && \ryy>\tiplength}
		% \ifnum\pgfmathresult=1
		% 	\draw[\ldraw, {Latex[length=\lyy]}-{Latex[length=\tiplength]},line width=\llinewidth] (\tlx,\by) -- (\tlx, \tly) -- (\trx,\try) -- (\trx,\by);
		% \fi
		% \pgfmathparse{\lyy<=\tiplength/4 && \ryy>\tiplength}
		% \ifnum\pgfmathresult=1
		% 	\draw[\ldraw, -{Latex[length=\tiplength]},line width=\llinewidth] (\tlx,\by) -- (\tlx, \tly) -- (\trx,\try) -- (\trx,\by);
		% \fi
		% \pgfmathparse{\ryy<\tiplength && \ryy>\tiplength/4 && \lyy>\tiplength}
		% \ifnum\pgfmathresult=1
		% 	\draw[\ldraw, {Latex[length=\tiplength]}-{Latex[length=\ryy]},line width=\llinewidth] (\tlx,\by) -- (\tlx, \tly) -- (\trx,\try) -- (\trx,\by);
		% \fi
		% \pgfmathparse{\ryy<=\tiplength/4 && \lyy>\tiplength}
		% \ifnum\pgfmathresult=1
		% 	\draw[\ldraw, {Latex[length=\tiplength]}-,line width=\llinewidth] (\tlx,\by) -- (\tlx, \tly) -- (\trx,\try) -- (\trx,\by);
		% \fi
	
		\begin{scope}
			\clip (\tlx,\tly) -- (\trx, \try) -- (\trx,\by) -- (\tlx,\by) ;
			\foreach \i in {1,...,\numarrows}
			\def\x{\i*\xspacing}
			\def\y{\i*\yspacing}
			\draw[\ldraw, -{Latex[length=\tiplength]}, line width = \llinewidth pt] ($ (\tlx,\tly)+(\x,\y) $) -- ($ (\tlx,\by)+(\x,0) $);
		\end{scope}

		\pgfmathparse{\ryy>\tiplength}
		\ifnum\pgfmathresult=1
			\draw[\ldraw, -{Latex[length=\tiplength]},line width=\llinewidth] (\trx,\try) -- (\trx,\by);
		\fi

		\pgfmathparse{\lyy>\tiplength}
		\ifnum\pgfmathresult=1
			\draw[\ldraw, -{Latex[length=\tiplength]},line width=\llinewidth] (\tlx,\tly) -- (\tlx,\by);
		\fi

		
	\end{scope}
	
}
