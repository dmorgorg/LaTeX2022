\documentclass[8pt,xcolor={svgnames, x11names}]{beamer}
\usefonttheme{professionalfonts}
\usepackage{amsmath}

\definecolor{miscstructurecolor}{rgb}{0.65,0.65,0.65}

% \usepackage{enumitem} % incompatible with Beamer?
\everymath{\displaystyle}

\usepackage{tikz}
\usepackage{pgfmath}
\usetikzlibrary{math, calc, intersections, arrows.meta}

% counter for resuming enumerated list numbers
\newcounter{resumeenumi}
\newcommand{\suspend}{\setcounter{resumeenumi}{\theenumi}}
\newcommand{\resume}{\setcounter{enumi}{\theresumeenumi}}

\newcounter{saveenumi}
\newcommand{\seti}{\setcounter{saveenumi}{\value{enumi}}}
\newcommand{\conti}{\setcounter{enumi}{\value{saveenumi}}}
\resetcounteronoverlays{saveenumi}


% \newcounter{myexercisecounter}
% \newcommand{\suspendExerciseCounter}{\setcounter{myexercisecounter}{\theenumi}}
% \newcommand{\resumeExerciseCounter}{\setcounter{enumi}{\themyexercisecounter}}

\newcommand\lb{\linebreak}
\newcommand\pars{\par\smallskip}
\newcommand\parm{\par\medskip}
\newcommand\parb{\par\bigskip}

%left flushed minipage
\newcommand{\minit}[2][0.8]{
	\begin{minipage}[t]{#1\columnwidth}
		\raggedright
		#2
	\end{minipage}
}

%left flushed minipage
\newcommand{\mini}[2][0.8]{
	\begin{minipage}[c]{#1\columnwidth}
		\raggedright
		#2
	\end{minipage}
}

% centered minipage with text \raggedright
%\cmini[width]{content}
\newcommand{\cmini}[2][0.8]{
	\begin{center}
		\begin{minipage}{#1\columnwidth}
			\raggedright
			#2
		\end{minipage}
	\end{center}
}

\newcommand{\cfig}[2][1]{% centred, scaled graphic
	\begin{center}
		\includegraphics[scale=#1]{#2}
	\end{center}
}
% figure with tight border for photos
% \cfigb[saitMaroon]{borderwidth with unit}{scale}{image}
\newcommand{\cfigb}[4][structure]{
	% \usepackage{adjustbox}
	\setlength{\fboxrule}{1pt}
	\begin{center}
		\includegraphics[scale=#3, cframe= #1 #2]{#4}
	\end{center}
}
\newcommand{\imgbox}[3]{
	% \setlength{\fboxsep}{12pt}
	\includegraphics[scale=#1, cframe= structure #3]{#2}
}

% get x and y coordinates from a tikz coordinate
%\gettikzxy{A}{\ax}{\ay}
\makeatletter
\providecommand{\gettikzxy}[3]{%
	\tikz@scan@one@point\pgfutil@firstofone#1\relax
	\edef#2{\the\pgf@x}%
	\edef#3{\the\pgf@y}%
}
\makeatother

\definecolor{staticsRed}{RGB}{128, 30, 45}
\definecolor{mucus}{rgb}{0.55,0.53,0.31}
\definecolor{myGreen}{RGB}{0,150,0}
\definecolor{saitPurple}{RGB}{112,40,119}
\definecolor{saitDeepBlue}{RGB}{0, 99, 167}
\definecolor{saitBlue}{rgb}{0, 0.59, 0.85}

\definecolor{DarkKhakiMid}{RGB}{235, 228, 134}


%  \definecolor{saitRed}{RGB}{224,38,37} 

%  \definecolor{khaki}{RGB}{190, 183, 107}
%\definecolor{philpotBlue}{RGB}{13,69,120}
% \definecolor{dRed}{rgb}{0.7,0,0}
% \definecolor{blueGrey}{rgb}{0.4,0.48,0.53}
% \definecolor{white}{rgb}{1,1,1}
% \definecolor{dkgreen}{rgb}{0,0.5,0}
% \definecolor{greenyellow}{rgb}{0.9,0.9,0.5}
% \definecolor{flesh}{rgb}{1, 0.95, 0.8}
% \definecolor{wheat}{rgb}{.96, .87, .70}
% \definecolor{oldlace}{rgb}{.992, .96187, .902}
% \definecolor{snow}{rgb}{1, .98, .98}
% \definecolor{ghostwhite}{rgb}{.973, .973, 1}
% \definecolor{cornsilk}{rgb}{1, .973, .863}
% \definecolor{honeydew}{rgb}{.941, 1, .941}
% \definecolor{lavenderdark}{rgb}{.8, .8, .9529411}
% \definecolor{lavender}{rgb}{.902, .902, .980}
% \definecolor{lightblue}{rgb}{.8, .8, .95}
\definecolor{lightgray}{rgb}{.827, .827, .827}
% \definecolor{lightsteelblue}{rgb}{.690, .769, .871}
% \definecolor{lightturquoise}{rgb}{.686, .933, .933}
% \definecolor{darkgreen}{rgb}{.0, .392, .0}
% \definecolor{yellowgreen}{rgb}{.604, .804, .196}
% \definecolor{vlightblue}{rgb}{.88, .85, .95}
% \definecolor{khaki}{rgb}{.741, .718, .42}
% \definecolor{lightkhaki}{rgb}{1, .96, .7}
% \definecolor{almostwhite}{rgb}{1,.95,1}
% \definecolor{facegreen}{rgb}{.45, .5, .2}
% \definecolor{llllBlueGrey}{rgb}{0.8,0.96,1}
% \definecolor{lllBlueGrey}{rgb}{0.69,0.83,0.92}
% \definecolor{llBlueGrey}{rgb}{0.58,0.69,0.76}
% \definecolor{lBlueGrey}{rgb}{0.48,0.58,0.64}
% \definecolor{blueGrey}{rgb}{0.4,0.48,0.53}
% \definecolor{dBlueGrey}{rgb}{0.33,0.4,0.44}
% \definecolor{ddBlueGrey}{rgb}{0.28,0.33,0.37}
% \definecolor{dddBlueGrey}{rgb}{0.23,0.28,0.31}
% \definecolor{almostBlue}{rgb}{0.985,0.985,1}
% \definecolor{almostGreen}{rgb}{0.9,0.97,0.9}

% \definecolor{almostRed}{rgb}{0.95,0.875,0.8}
%  \definecolor{headerGrey}{RGB}{128,128,128}
%  \definecolor{headerGray}{RGB}{128,128,128}
% \definecolor{dHeaderGrey}{RGB}{96,96,96}
% \definecolor{ddHeaderGrey}{RGB}{64,64,64}
% \definecolor{dddHeaderGrey}{RGB}{32,32,32}
% \definecolor{lHeaderGrey}{RGB}{160,160,160}
% \definecolor{llHeaderGrey}{RGB}{192,192,192}
% \definecolor{lllHeaderGrey}{RGB}{224,224,224}
% \definecolor{philpotBlue}{RGB}{13,69,120}
% \definecolor{drabGreen}{RGB}{156,143,87}
% \definecolor{ground}{RGB}{153,153,51}
% \definecolor{gridLight}{rgb}{0.85,0.85,0.85}
% \definecolor{darkGreen}{rgb}{0,0.5,0}


\usepackage[absolute,overlay]{textpos}
\setlength{\TPHorizModule}{1.0cm}
\setlength{\TPVertModule}{\TPHorizModule}
\textblockorigin{0.0cm}{0.0cm}  %start all at upper left corner


\usetheme{Antibes}
\usecolortheme[named=miscstructurecolor]{structure}
\setbeamertemplate{items}[triangle]
\setbeamertemplate{blocks}[rounded][shadow=false]
\setbeamertemplate{headline}{\vspace{.1cm}}
\setbeamertemplate{navigation symbols}{} % empty braces suppresses all navigation symbols
\setbeamertemplate{footline}{
	\hfill
	\insertshorttitle
	\quad
	\insertsection
	\quad
	\insertsubsection
	\quad
	\insertframenumber/\inserttotalframenumber
	\quad{ }
	\vspace{0.125cm}
}
\addtobeamertemplate{footline}{\hypersetup{linkcolor=black}}{}
\setbeamertemplate{navigation symbols}{} % empty braces suppresses all navigation symbols
\setbeamercolor{frametitle}{fg=white}

\usepackage{hyperref} % hyperref usually loaded last but automatically by Beamer
\hypersetup{
	colorlinks,
	citecolor=red,
	filecolor=orange,
	bookmarksopen=true,
	linkcolor=staticsstructurecolor, % table of contents
	urlcolor=blue
}
\usepackage{bookmark}

% various hacks to get around (apparent?) Beamer titlepage constraints
\title[\color{black}Miscellaneous Nerdery]{\color{black}\Huge Nerdstuff}
\subtitle{} % i.e., a blank line
\institute{\normalsize Source code at: {\small\url{https://github.com/dmorgorg/LaTeX/blob/master/misc/misc.pdf}}}
\author{} % another blanky
\date{\small Last updated on \today}


\everymath{\displaystyle}

\begin{document}
\small
\begin{frame}[plain]
	\titlepage
\end{frame}

\section{Geometry}

%%%%%%%%%%%%%%%%%%%%%%%%%%%%%%%%%%%%%%%%%%%%%%%%%%%%%%%%%%%%%%%%%%%%%%%%%%%%%%%%%%%%%%%%%%%%%%%%%%%%%%%%%%%%%%%%%%%%%%%%

\subsection{Construct a 30 Degree Angle}



\begin{frame}{Geometry :: How To Construct A $30^\circ$ Angle With A Pair Of Compasses}

	\setbeamercolor{normal text}{fg=gray,bg=}
	% for highlighting current enumeration item
	\setbeamercolor{alerted text}{fg=black,bg=}
	\usebeamercolor{normal text}

	\begin{textblock*}{0.8\textwidth}(2cm, 1.25cm)
		\pgfsetroundjoin
		\pgfsetroundcap
		\centering
		% \small
		\tikz{
			\begin{scope}[scale=0.85]
				\coordinate (A) at (0,0);
				\coordinate (B) at (4,0);
				\coordinate (C) at (2,3.464);
				\coordinate (D) at ($ (B)!.5!(C) $);
				
				\uncover<2-5>{
					\draw[gray] ($(A)+(-1,0)$) -- ($(B)+(1,0)$);
				}
				\uncover<2-4>{
					\filldraw[fill=white, draw=black] (A) circle (0.3mm);					
				}
				\uncover<2-8>{
					\node[above left] at (A){$A$};
				}
				\uncover<3-8>{
					\node[above right] at (B){$B$};					
				}
				\uncover<4-7>{
					\node[above] at (C){$C$};
				}
				\uncover<3-5>{
					\draw[gray] (B) arc (0:70:4cm);					
					\draw[gray] (B) arc (0:-5:4cm);
				}
				
				\uncover<3-4>{
					\filldraw[fill=white, draw=black] (B) circle (0.3mm);		
				}
				\uncover<3>{						
					\draw[red, thin, latex-latex] (A) -- node[above, sloped]{$r$}+(30:4cm);		
				}
				\uncover<4-5>{
					\draw[gray] (A) arc (180:110:4cm);					
					\draw[gray] (A) arc (180:185:4cm);
				}
				\uncover<4>{
					\filldraw[fill=white, draw=black] (C) circle (0.3mm);
					\filldraw[fill=white, draw=black] (A) circle (0.3mm);	
				}
				\uncover<4>{						
					\draw[red, thin, latex-latex] (B) -- node[above, sloped]{$r$}+(150:4cm);		
				}
				\uncover<5>{
					\draw[thick] (A)--node[above]{$r$}(B)--node[above, sloped]{$r$}(C)--node[above, sloped]{$r$}cycle;
					\node at ($ (A)+(25:0.5cm) $){$60^\circ $};
					\node at ($ (B)+(155:0.5cm) $){$60^\circ $};
					\node at ($ (C)+(270:0.625cm) $){$60^\circ $};
				}
				\uncover<6-7>{
					\draw[thick] (A)--(B)--(C)--cycle;
					\draw[gray] ($ (C)!0.25!(B)$) arc (120:175:3cm);
					\draw[gray] ($ (C)!0.25!(B)$) arc (120:65:3cm);
					\draw[gray] ($ (B)!0.25!(C)$) arc (-60:-115:3cm);
					\draw[gray] ($ (B)!0.25!(C)$) arc (-60:-5:3cm);
				}
				\uncover<7>{
					\draw[gray] ($(D)+(210:4cm) $) -- ($(D)+(30:3cm)$);					
					\filldraw[fill=white, draw=black] (D) circle (0.3mm);
				}
				\uncover<7-8>{
					\node[above, xshift=1mm, yshift=0.5mm] at (D){$D$};
				}
				\uncover<8>{
					\node[above, gray] at (C){$C$};
					\draw[gray] (A)--(B)--(C)--cycle;
					\draw[thin] ($(D)+(-60:0.2)$)-- ++(210:0.2)-- ++(120:0.2);
					\draw[myGreen, very thick] (D)--(A)--(B)--cycle;
					\node at ($ (B)+(155:0.5cm) $){$60^\circ $};
					\node at ($ (A)+(13.75:0.875cm) $){$30^\circ $};
				}
						
		 \end{scope}
		} % end \tikz
	\end{textblock*}	

	\begin{textblock*}{0.55\textwidth}(0.5cm, 5.5cm)
		
		\begin{enumerate}
			\item<1-8 | alert@1> To do this, you will need a pair of compasses (sometimes known, incorrectly, as a compass: whatever you call it, you need the device that draws circles or arcs) and a sheet of paper to work on.
			\item<2-8 | alert@2> Draw a horizontal line close to the bottom of a sheet of paper and mark a point $A$ near the left end of the line.
			\item<3-8 | alert@3> Using the pair of compasses, draw an arc with centre $A$ and radius $r$ as shown. Mark the intersection of the arc with the line as point $B$.
			\suspend
		\end{enumerate}
	
	\end{textblock*}

	\begin{textblock*}{0.55\textwidth}(6.5cm, 5.5cm)
		
		\begin{enumerate}
			\resume
			\item<4-8 | alert@4> Keeping the radius at $r$, draw an arc with centre $B$ as shown. Mark the intersection of the two arcs as $C$.
			\item<5-8 | alert@5> $\triangle ABC$ is equilateral with sides of length $r$.
			\item<6-8 | alert@6> Draw arcs centred at $B$ and $C$, with radius $r'\approx 0.75r$, as shown.
			\item<7-8 | alert@7> Draw a line between the intersection of these two arcs. This line bisects $BC$ at $D$. It also passes through $A$, bisecting $\angle BAC$.
			\item<8 | alert@8> $\angle BAD = 30^\circ$, as required.
			\suspend
		\end{enumerate}
	
	\end{textblock*}
	
\end{frame}

%%%%%%%%%%%%%%%%%%%%%%%%%%%%%%%%%%%%%%%%%%%%%%%%%%%%%%%%%%%%%%%%%%%%%%%%%%%%%%%%%%%%%%%%%%%%%%%%%%%%%%%%%%%%%%%%%%%%%%%%
\subsection{Calculate Belt-Length}


\begin{frame}{Geometry :: Belt-Length}
	\setbeamercolor{normal text}{fg=gray,bg=}
	% for highlighting current enumeration item
	\setbeamercolor{alerted text}{fg=black,bg=}
	\usebeamercolor{normal text}

	\begin{textblock*}{0.6\textwidth}(1cm, 1.25cm)
		\pgfsetroundjoin
		\pgfsetroundcap
		\centering
		% \small
		\tikz{
			\begin{scope}[scale=0.65]
				\def\rA{1}
				\def\rB{2.5}
				\def\d{6}
				\coordinate (A) at (0,0);
				\coordinate (B) at (\d,0);
				\coordinate (AA) at (-\rA,0);
				\coordinate (Bprime) at ($(B)+(\rB,0)$);
				\filldraw[fill=Gainsboro] (A) circle [radius=\rA];
				\filldraw[fill=Gainsboro, name path=bigly] (B) circle [radius=\rB];

				\fill (B) circle (.6mm) node[above, xshift=1.5mm]{$B$};

				\only<1-11>{
					\draw[red, latex-latex] (A)--node[fill=Gainsboro,inner sep=0.1mm]{$r$}+(225:\rA);
					\draw[red, latex-latex] (B)--node[pos=0.77, fill=Gainsboro,inner sep=0.25mm]{$R$}+(-45:\rB);
					\fill (A) circle [radius=.6mm] node[below, xshift=1.5mm]{$A$};
				}

				\only<1>{
					% some slight 'alterations' in later decimal digits to close the path
					\draw[very thick, ForestGreen] ($(A)+(-1,0)$) arc (180:105:\rA cm) -- ++(14.5:5.815) arc (104.5:-104.5:\rB cm)--++(165.5:5.806)arc(255.5:180:\rA);
				}


				\tikzmath{
					\r2=\rA-\rB;
				}

				\only<1-4>{
					\draw[red, latex-latex] (A)--node[fill=white,inner sep=0.5mm]{$d$}(B);
				}
				\onslide<2-8>{
					\draw[name path=concentric, blue] (B) circle [radius=\r2];
				}
				\onslide<3-8>{
					\draw[name path=semi, blue] ($ (A)!.5!(B)$) circle [radius=3];
				}
				\only<4>{
					\fill [name intersections={of=concentric and semi}]
					(intersection-1) circle (2pt) node[below, xshift=1mm] {$D$};
				}
				\onslide<4>{
					% /tikz/intersection/by=<hcomma-separated list> in pgf/tikz manual
					\fill [name intersections={of=concentric and semi, by={D, C}}]
					(intersection-2) circle (2pt) node[above, xshift=1mm] {$C$};
				}
				\onslide<5-8>{
					\filldraw[ thick, fill=green!25!white, draw=black] (A)--(B)--(C)--cycle;
					\fill (C) circle (0.6mm) node[above, xshift=1mm] {$C$};
				}
				\onslide<5-11>{
					\node at (B)[xshift=-1.5mm, yshift=1.3mm] {$\theta$};
					\node[below] at ($ (A)!0.5!(B) $) {$d$};
				}
				\onslide<6-8>{
					\draw[name path=CCprime, thick] (B)--($ (B)!1.67!(C) $);
					\fill [name intersections={of=CCprime and bigly, by={Cprime, Dprime}}];
					% (Cprime) circle (0.6mm) node[above, xshift=1mm] {$C'$};
				}
				\onslide<6-11>{
					\fill (Cprime) circle (0.6mm)node[above, xshift=1mm] {$C'$};
				}
				\onslide<7-8>{
					\coordinate(deltaBC) at ($ (C)-(B) $);
					\gettikzxy{(deltaBC)}{\dx}{\dy}
					\pgfmathparse{\dx==0}
					\ifnum\pgfmathresult=1 % \dx == 0
						\pgfmathsetmacro{\rot}{\dy > 0 ? 90 : -90}
					\else
						\pgfmathsetmacro{\rot}{\dx > 0 ? atan(\dy / \dx) : 180 + atan(\dy / \dx)}
					\fi
					\coordinate (Aprime) at (\rot:\rA);
					\fill (Aprime) circle (0.6mm) node[above, xshift=1mm] {$A'$};
					\draw[thick] (A) -- (Aprime);
				}
				\onslide<7-11>{
					\draw (B)--(Cprime);
					\draw (A)--(Aprime);
				}
				\onslide<8-11>{
					\fill (Aprime) circle (0.65mm) node[above, xshift=1mm] {$A'$};
					\draw[very thick, ForestGreen] (Aprime)--(Cprime);
				}
				\onslide<9-11>{
					\fill (AA) circle (0.65mm) node[left] {$D$};
					\fill (Bprime) circle (0.65mm) node[right] {$E$};
					\draw (AA) -- (Bprime);
					\node at (A)[xshift=-1.5mm, yshift=1.3mm] {$\theta$};
				}
				\onslide<10-11>{
					\pgfsetbuttcap
					\draw[very thick, ForestGreen] (AA) arc (180:105:\rA cm);
					\draw[very thick, ForestGreen] (Cprime) arc (104.5:0:\rB cm);
				}
				\onslide<12>{
					\fill (A) circle [radius=.6mm] node[above, xshift=-1.5mm]{$A$};
					\footnotesize
					\draw[red, latex-latex] (A)--node[pos=0.37,fill=white]{$30.0\,$cm}(B);
					% some slight 'alterations' in later decimal digits to close the path
					\draw[very thick, ForestGreen] ($(A)+(-1,0)$) arc (180:105:\rA cm) -- ++(14.5:5.815) arc (104.5:-104.5:\rB cm)--++(165.5:5.806)arc(255.5:180:\rA);
					\draw[red, latex-latex] (A)--node[rotate=-45,fill=Gainsboro,inner sep=0.35mm]{$6.00\,$cm}+(225:\rA);
					\draw[red, latex-latex] (B)--node[rotate=45,fill=Gainsboro,inner sep=0.35mm]{$15.00\,$cm}+(-45:\rB);
				}
			\end{scope}
		}
	\end{textblock*}

	\begin{textblock*}{0.4\textwidth}(8.5cm, 2.5cm)
		\mini{
			Two pulleys, centred at $A$ and $B$, have radii $r$ and $R$. The distance from $A$ to $B$ is $d$. \parm
			Determine the length of the belt required to go round both pulleys.
		}
	\end{textblock*}

	\begin{textblock*}{1\textwidth}(1cm, 5.5cm)
		\minit[0.475]{
			\begin{enumerate}
				\item<2-5 | alert@2> Construct a circle, diameter $R\!-\!r$, centred at $B$.
				\item<3-5 | alert@3> Construct a circle with diameter $AB$.
				\item<4-5 | alert@4> These two circles intersect at $C$ and $D$. Due to the horizontal axis of symmetry through $A$ and $B$, we only need perform calculations on one half of the system.
				      \suspend
			\end{enumerate}
		}
		\hfill
		\minit[0.475]{
			\begin{enumerate}
				\resume
				\item<5-8 | alert@5> Consider $\triangle ABC$: $\angle ACB=90^\circ$ since it is an angle inscribed in a semicircle. Then:
				      \begin{align*}
					      AC     & = \sqrt{d^2-(R-r)^2}                                 \\
					      \theta & = \sin^{-1}\left(\frac{\sqrt{d^2-(R-r)^2}}{d}\right)
				      \end{align*}
				      \suspend
			\end{enumerate}
		}
	\end{textblock*}

	\begin{textblock*}{1\textwidth}(1cm, 5.5cm)
		\minit[0.475]{
			\begin{enumerate}
				\resume
				\item<6-10 | alert@6> Extend line $BC$ to $C'$ on the circumference of pulley $B$. $CC'$ has length $r$.
				\item<7-10 | alert@7> Draw $AA'$, of length $r$ and parallel to $CC'$, as shown.
				\item<8-10 | alert@8> Draw $A'\!C'$: $A'\!C'\!C\!A$ is a rectangle so
				      \[ A'\!C' = AC = \sqrt{d^2-(R-r)^2} \]
				      \suspend
			\end{enumerate}
		}
		\hfill
		\minit[0.475]{
			\begin{enumerate}
				\resume
				\item<9-11 | alert@9> $A'\!C'$ is the (top) part of the belt that is tangential to the pulleys at $A'$ and $C'$. We now need to find the arc-lengths from $D$ to $A'$ and from $C'$ to $E$.
				\item<10-11 | alert@10>The angles ($\theta$ and $\pi\!-\!\theta$) that these arcs subtend at the pulley centres, with  the radius of each pulley, are used to determine the arc-lengths ($\theta$ in radians):
				      $$ D\!A'=r\theta \text{ and } C'\!D=R(\pi\!-\!\theta) $$
				      \suspend
			\end{enumerate}
		}
	\end{textblock*}

	\begin{textblock*}{1\textwidth}(1cm, 5.5cm)
		\minit[0.475]{
			\begin{enumerate}
				\resume
				\item<11 | alert@11> {
						      Belt-length:
						      \begin{align*}
							             & =2\left(DA'+A'C'+C'E\right)                                     \\
							             & =2\left(r\theta + \sqrt{d^2-(R-r)^2} + R(\pi\!-\!\theta)\right) \\
							      \intertext{ where }
							      \theta & = \sin^{-1}\left(\frac{\sqrt{d^2-(R-r)^2}}{d}\right)
						      \end{align*}
					      }
			\end{enumerate}
		}
	\end{textblock*}

	\begin{textblock*}{1\textwidth}(1cm, 5.5cm)
		\mini[1]{
			\only<12>{ \underline{Example}:
				\begin{gather*}
					\theta = \sin^{-1}\left(\frac{\sqrt{d^2-(R-r)^2}}{d}\right) = \sin^{-1}\left( \frac{\sqrt{6.00^2-1.50^2}}{6.00} \right)= 1.3181\  \text{(radians)}\\
					\text{B-L} = 2\left( 6.00\times1.3181+\sqrt{6.00^2-1.50^2}+15.00\times(\pi-1.3181) \right) = 82.141
				\end{gather*}
				\centering\parb
				\underline{The belt length is $82.1\,$cm.}
			}
		}
	\end{textblock*}
\end{frame}

%%%%%%%%%%%%%%%%%%%%%%%%%%%%%%%%%%%%%%%%%%%%%%%%%%%%%%%%%%%%%%%%%%%%%%%%%%%%%%%%%%%%%%%%%%%%%%%%%%%%%%%%%%%%%%%%%%%%%%%%

\section{Web}
\begin{frame}{Web :: Dynamic font sizes with CSS}
	\only<1-5>{
	\begin{itemize}
		\item<+-> Responsive websites usually require font sizes that change with device (or browser window) width. A font size of 20px that works well on a large monitor is unlikely to be suitable for a smaller tablet or a phone. 
		\item<+-> With media queries, you can set a different font size for each range of device sizes; this is perfectly adequate in many cases. It does have the disadvantage that when resizing a browser, the user will see the font size jumping from, for example, 18px to 16px to 14px. (But it's usually only designers who spend too much time resizing windows.) 
		\item<+-> For a more fluid result, viewport units are useful (where {\ttfamily 1vw = 1/100} of the window width). So, for example, if you set your css to \textcolor{red}{\ttfamily font-size:2vw;} and your phone is 400px wide, the font will be 8px. If your window width is 1200px, font size will be 24px. 
		\item<+-> Using viewport units alone may tend to make the fonts too small on small screens and too large on large screens; one can introduce some fixed sizes as well:\newline \textcolor{red}{\ttfamily font-size:calc(9px + 1vw);} sets font size at 13px for phone width of 400px and font size of 21px for window size of 1200px. \pars Still not exactly what you want? Maybe \textcolor{red}{\ttfamily font-size:calc(4px + 1.5vw);}?
		\item<+-> For more precise control, without trial and error that rapidly becomes frustrating, I turned to this excellent \href{https://css-tricks.com/snippets/css/fluid-typography/}{CSS-tricks} page showing examples such as
		\begin{center}
			 \textcolor{red}{\ttfamily font-size: calc(16px + 6 * ((100vw - 320px) / 680));}
		\end{center}
		What are these magic numbers? At the moment, I can tell that, at a screen width of 320px, the font size is 16px. Font size increases smoothly until, at a screen width of 1000px, the font size of 22px. But I probably won't remember how to figure that out next week! The \href{https://css-tricks.com/snippets/css/fluid-typography/}{CSS-tricks} page doesn't show how these numbers are derived\ldots \pars \ldots but it's just some (relatively) simple high-school math. All you need to recall from high-school is the equation of a line in the form: $\frac{y-y_1}{x-x_1}=\frac{y_2-y_1}{x_2-x_1}$
	\end{itemize}
	}
	\only<6->{
		\begin{textblock*}{1\textwidth}(1cm, 1cm)
			\cmini{
				\tikz{
					\coordinate (0) at (0,0);
					\coordinate (X) at (6,0);
					\coordinate (Y) at (0,3.5);
					\coordinate (min) at (1,1.25);
					\coordinate (max) at (5,2.5);

					\draw[-to](0)--node[below, pos=0.9, yshift=-0.4cm]{\bf view width}(X);
					\draw[-to](0)--node[left,pos=0.9, xshift=-0.15cm]{\bf font size}(Y);
					\node[below left] at (0){0};

					\gettikzxy{(min)}{\minx}{\miny}
					\gettikzxy{(max)}{\maxx}{\maxy}
					\gettikzxy{(X)}{\xx}{\xy}

					\only<6-8,16,17>{
						\draw[very thick, red](0,\miny)--(\minx,\miny)--(max)--(\xx,\maxy);
					}
					
					\only<6-7,16,17>{						
						\node[left] at (0,\miny){10px};
						\draw[dotted](max)--(0,\maxy)node[left]{20px};
						\draw[dotted](min)--(\minx,0)node[below]{300px};
						\draw[dotted](max)--(\maxx,0)node[below]{1200px};	
					}	
					\only<8>{
						\draw[dotted](min)--(\minx,0)node[below]{$x_1$};
						\node[yshift=-0.2cm,fill=white, inner sep=0.4mm] at (min) {$(x_1,y_1)$};
					}
					\only<8-14>{
						\node[left] at (0,\miny){$y_1$};
						\draw[dotted](max)--(0,\maxy)node[left]{$y_2$};						
						\draw[dotted](max)--(\maxx,0)node[below]{$x_2$};							
						\node[yshift=0.2cm,fill=white, inner sep=0.4mm] at (max) {$(x_2,y_2)$};	
					}	
					\only<9-14>{
						\draw[dotted](\minx,\maxy)--(\minx,0)node[below]{$x_1$};
						\draw[dotted](\maxx,\miny)--(0,\miny);
						\node[yshift=-0.2cm,fill=white, inner sep=0.4mm] at (min) {$(x_1,y_1)$};
						\draw[very thick, red](min)--(max);
					}
					\only<9-14>{
						\coordinate (P) at ($(min)!0.6!(max)$);
						\gettikzxy{(P)}{\px}{\py}
						\draw[thick, green!50!black] (P)--(\px,0)node[below, black]{\large $ x$};
						\draw[thick, green!50!black] (P)--(0,\py)node[left, black]{\large $ y$};
					}
					\only<15>{
						\node[yshift=-0.2cm,fill=white, inner sep=0.4mm] at (min) {$(300,10)$};
						\node[yshift=0.2cm,fill=white, inner sep=0.4mm] at (max) {$(1200,20)$};	
						\draw[very thick, red](min)--(max);
					}			
				}
			}
		\end{textblock*}
	}
	\begin{textblock*}{1\textwidth}(1cm, 5.75cm)	
		\only<6-8>{
			\begin{enumerate}		
				\item<6-8> We can graph our desired font size against view width as shown. In this case: view widths less than 300px have a font size of 10px; font sizes grow uniformly from 10px at 300px view width to a font-size 20px at 1200px window width; and, for window sizes over 1200px, the font size remains 20px. \textcolor{red}{How do we achieve this with CSS?}				
				\item<7-8> The constant font sizes below 300px and above 1200px can be easily handled with media queries; we'll come back to them later. What is more interesting is the uniformly increasing font size calculation between view widths of 300px and 1200px. 
				\item<8> Instead of using the fixed numbers, we'll generalise and use variables so we can easily adjust our formula for different required values.
				\suspend							
			\end{enumerate}
		}
		\only<9-12>{
			\begin{enumerate}	
				\resume			
				\item<9-12> For now, just focus on the sloped line as a function from view width $x$ to font size $y$.
				\item<10-12> The equation of that line is given by:
					\begin{align*}
						\uncover<10-12>{\frac{y-y_1}{x-x_1} &= \frac{y_2-y_1}{x_2-x_1} \\}
						\uncover<11-12>{\Rightarrow y-y_1 &= (x-x_1)\cdot\frac{y_2-y_1}{x_2-x_1}\\}
						\uncover<12>{\Rightarrow y &= y_1+(x-x_1)\cdot\frac{y_2-y_1}{x_2-x_1}}
					\end{align*}								
				% \suspend % align environment seems to break this					
			\end{enumerate}			
		}
		\only<13-15>{
			\begin{enumerate}
				\setcounter{enumi}{5}		% first item will be 6				
				\item<13-> We have font size $ = y_1+(x-x_1)\cdot\frac{y_2-y_1}{x_2-x_1}$ where $x_1$, $y_1$, $x_2$ and $y_2$ are numbers chosen for our particular design and $x$ is view width. Of course, CSS does not understand $x$ but view width can be represented by 100vw.
				\item<14-> Thus,  \textcolor{red}{\ttfamily font-size:calc($y_1$ + (100vw - $x_1$)*($y_2-y_1$)/($x_2-x_1$));}	
				\item<15-> From our previous example:
				\begin{center}
					\textcolor{red}{\ttfamily font-size:calc(10px + (100vw - 300px)*(20px-10px)/(1200px-300px));}
				\end{center}					
				or, more concisely:
				\begin{center}
					\textcolor{red}{\ttfamily font-size: calc(10px + 10 * (100vw - 300px)/900);}
				\end{center}
				
			\end{enumerate}			
		}
		\only<16>{
			\begin{enumerate}
				\setcounter{enumi}{8}		% first item will be 9				
				\item CSS for the complete range of view widths: \cmini{
				
					\textcolor{red}{\ttfamily @media screen and (min-width:1200px) \{ \newline
					html \{ font-size:20px; \} \newline\} } \pars

					\textcolor{red}{\ttfamily @media screen and (max-width:1200px) \{ \newline
						html \{ font-size: calc(10px + (100vw - 300px)/90); \} \newline \} } \pars

					\textcolor{red}{\ttfamily @media screen and (max-width:300px) \{ \newline
						html \{ font-size:10px; \} \newline\} } 
				}					 			
			\end{enumerate}			
		}
		\only<17>{
			\begin{enumerate}
				\setcounter{enumi}{9}		% first item will be 9				
				\item For ease of editing, using SASS variables for min-font, min-width, max-font, max-width is a better solution. Or write a mixin... 
			\end{enumerate}			
		}

	\end{textblock*}

% 	@media screen and (min-width: 1000px) {
%   html {
%     font-size: 22px;
%   }
% }
	

	


		

	
\end{frame}
\end{document}
